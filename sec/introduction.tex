%!TEX root = ../per_st_tk.tex

\begin{frame}{Today's goal}
	\pause
	Introduce \colorit{Steenrod barcodes}, new invariants for TDA,
	\pause
	which appear in real-data, are robust to noise, and computable using \colorit{\texttt{steenroder}}, a ready-to-use \texttt{Python} tool.

	\pause\medskip
	\colorit{Example}
	\vskip-15pt
	\begin{figure}
		\centering
		\begin{subfigure}[b]{0.49\textwidth}
			\centering
			\includegraphics[width=\textwidth]{aux/s2_s4.pdf}
			\caption{$\mathrm C\,\Sigma(S^2 \vee S^4)$}
			\label{f:s2_s4}
		\end{subfigure}
		\begin{subfigure}[b]{0.49\textwidth}
			\centering
			\includegraphics[width=\textwidth]{aux/cp2.pdf}
			\caption{$\mathrm C\,\Sigma\,\bC\rP^2$}
			\label{f:cp2}
		\end{subfigure}
	\end{figure}
\end{frame}

\begin{frame}{Viewpoint}
	\pause
	\colorit{A goal of algebraic topology} \\
	To construct invariants of spaces up to some notion of equivalence.

	\bigskip\pause
	\colorit{Today} \\
	CW complexes and homotopy equivalence.

	\bigskip\pause
	\colorit{A basic tension} \\
	Computability \colorit{vs} strength of invariants.

	\bigskip\pause
	\colorit{Example} \\
	Cohomology \colorit{vs} homotopy.

	\bigskip\pause
	\colorit{A more subtle one} \\
	Effectiveness \colorit{vs} functoriality of their constructions.

	\bigskip\pause
	\colorit{Example} \\
	Cohomology via chain complex \colorit{vs} maps to Eilenberg-Maclane spaces.
\end{frame}

\begin{frame}{Effectively defined cohomology}
	\pause
	\colorit{Poincar\'{e}'s idea} \\
	Break spaces into contractible combinatorial pieces: \\
	\begin{center}
		Simplices $\gsimplex^n$.
%		, cubes $\gcube^n$, ...
	\end{center}

%	\pause
%	\colorit{Kan's idea} \\
%	Replace spaces by functors with a geometric realization: \\
%	\begin{center}
%		Simplicial sets, cubical sets, ...
%	\end{center}

	\pause
	\colorit{Compute cohomology} \\
	Using a chain complex assembled from the standard chain complexes: \\
	\begin{center}
		$\gchains(\gsimplex^n)$.
%		, $\gchains(\gcube^n)$, ...
	\end{center}

	\pause
	\colorit{Enhance cohomology} \\
	With effective computations of finer invariants in cohomology by understanding

	\begin{center}
		$\gchains(\gsimplex^n) \to \gchains(\gsimplex^n) \ot \gchains(\gsimplex^n)$
	\end{center}
\end{frame}